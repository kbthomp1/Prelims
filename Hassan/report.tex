%        File: report.tex
%     Created: Sun Feb 21 08:00 AM 2016 E
% Last Change: Sun Feb 21 08:00 AM 2016 E
%
\documentclass[a4paper]{article}

\usepackage[a4paper, margin=1.25in]{geometry}

\usepackage{mathtools} 
\usepackage{amsmath}
\usepackage{amssymb}
\usepackage{newlfont}
\usepackage{caption}
\usepackage{subcaption}
\usepackage{titlesec}
\usepackage{graphicx}
\usepackage{empheq}

\newcommand{\eref}[1]{Eq.~(\ref{#1})}
\newcommand{\erefs}[2]{Eq.s~(\ref{#1}-\ref{#2})}
\newcommand{\pd}[2]{\frac{\partial #1}{\partial #2}}
\newcommand{\pnd}[3]{\frac{\partial^{#3} #1}{\partial #2^{#3}}}

\title{Written Preliminary Exam \#1 \break
       Prepared by Dr. Hassan}

\author{ Kyle B. Thompson }

\begin{document}
\maketitle

\begin{enumerate}
  \item Constant density flow and incompressible flow are two different things.
    Constant density implies that a fluid(s) density is constant, but capable of
    changing (through compression or expansion), whereas incompressible implies
    the fluid(s) density is incapable of changing but that the density in the
    flow is not necessarily constant.  A good example from the MAE 550 course to
    distinguish the two is stagnant air in a room, versus an oil/water mixture.
    Stagnant air in a room is at constant density, but is certainly capable of
    both compression or expansion; therefore, it is a constant density flow, but
    not incompressible.  An oil/water mixture is not constant density.  As
    illustrated in figure \ref{fig:incompressible},the water and oil are
    different fluids with different densities, but the fluids are incompressible
    because both the water and oil are not capable of compression or expansion
    (at least in a practical sense); therefore, it is not a constant density
    flow, but it is incompressible.

    \begin{figure}[h]
      \centering
      \includegraphics[width=0.8\textwidth,trim={0 2cm 0 2cm},clip]{oil_water_fig}
      \caption{Incompressible Fluid Flow Example}
      \label{fig:incompressible}
    \end{figure}

  \item The Pohlhausen method is typically defined as a velocity profile polynomial
    \begin{equation}
      \begin{gathered}
        \frac{u}{U_e} = a + b\eta + c\eta^2 + d\eta^3 + e\eta^4 \\
        \eta = \left( \frac{y}{\delta} \right)
      \end{gathered}
      \label{u-polynomial}
    \end{equation}
    where $u$ is the velocity, $U_e$ is the velocity at the edge of the
    boundary layer, and $\delta$ is the boundary thicknesss.  The polynomial
    coefficients are defined by the boundary conditions for $u = u(x,y)$, which
    Pohlhausen specified as
    \begin{align}
      u(x,0) &= 0 \\
      u(x,\delta) &= U_e \\
      \pd{u}{y}(x,\delta) &= 0 \\
      \pnd{u}{y}{2}(x,0) &=
      -\frac{U_e}{\nu}\pd{U_e}{x} \\
      \pnd{u}{y}{2}(x,\delta) &= 0 \label{bad-bc}
    \end{align}
    As discussed in MAE 589, the last boundary condition (\eref{bad-bc}) was a
    poor choice, because the fact that the first derivative of velocity, $\partial
    u/\partial y$, is zero at a point does not necessarily mean that the
    curvature at that point is also zero.  A suggested improvement would be to
    substitute that boundary condition for
    \begin{equation}
      \pnd{u}{y}{3}(x,0) = 0
      \label{improved-bc}
    \end{equation}
    which comes from differentiating the momentum equation
    \begin{equation}
      \begin{aligned}
        \pd{}{y}\left( u \pd{u}{x} + v\pd{u}{y}\right) &= \pd{}{y}\left(
        -\pd{p}{x} + \nu \pnd{u}{y}{2} \right) \\
        \pnd{u}{y}{3}(x,0) &= 0
      \end{aligned}
      \label{bc-deriv}
    \end{equation}
    With this new boundary condition, the polynomial coefficients can be derived
    as
    \begin{align}
      a &= 0 \\
      b &= \frac{4 + \lambda}{3} \\
      c &= -\frac{\lambda}{2} \\
      d &= 0 \\
      e &= \frac{\lambda - 2}{6}
      \label{coefs}
    \end{align}
    where $\lambda = \frac{\delta^2}{\nu}\pd{U_e}{x}$ is the Pohlhausen pressure
    gradient parameter.  The separation condition $\tau_w = \mu(\partial
    u/\partial y)_{y=0} = 0$ is used to determine the value of $\lambda$ at
    separation
    \begin{equation}
      \begin{aligned}
        \left( \pd{u}{y} \right)_{y=0} = 0 &= \frac{b}{\delta} \\
        0 &= \frac{4+\lambda}{3\delta} \\
        \lambda &= -4
      \end{aligned}
      \label{new-separation-point}
    \end{equation}
    The same procedure can be used with \eref{bad-bc} in place of
    \eref{improved-bc} to find that the original Pohlhausen method yields
    $\lambda = -12$ at separation.  The Falkner-Skan equations can be used to
    verify that \eref{improved-bc} does better at predicting separation for
    wedge flows, where $U_e = U_1 x^m$.  In the literature\cite{Shetz} it is
    reported that
    flow separates at $m = -0.0904$, and
    \begin{equation}
      \frac{\delta}{x}\sqrt{Re_x} = 7.1
      \label{bl-falkner-skan}
    \end{equation}
    at separation. This can be rearranged to give the exact value of the
    Pohlhausen pressure gradient parameter, $\lambda$, at separation
    \begin{equation}
      \begin{aligned}
        \left( \frac{\delta}{x}\sqrt{Re_x}\right)^2 &= \frac{\delta^2}{\nu}\frac{U_e}{x} \\
        &= \frac{\delta^2}{\nu}\frac{U_e}{x} \\
        &= \frac{\delta^2}{\nu}\pd{U_e}{x}\frac{1}{m} \\
        &= \frac{\lambda}{m}
      \end{aligned}
      \label{exact-lambda}
    \end{equation}
    Thus, the exact value of $\lambda$ at separation is
    \begin{equation}
      \begin{aligned}
        \lambda_{exact} &= m \left(\frac{\delta}{x}\sqrt{Re_x}\right)^2 \\
        &= (-0.0904)(7.1)^2 \\
        &= -4.56
      \end{aligned}
      \label{exact-lambda-value}
    \end{equation}
    It is therefore clear the Pohlhausen method with \eref{improved-bc},
    yielding $\lambda = -4$ instead of $\lambda = -12$ at separation, is an
    improvement over using the original boundary condition (\eref{bad-bc}).

  \item There has been significant progress in the recent decades in developing
    stress models that address second-order closure, where the unique terms in
    the reynolds stress tensor are modeled individually and the Boussinesq
    approximation is not used .  Wilcox\cite{wilcox} gives a good
    description of deficiencies of the Boussinesq approximation, stating that
    models based upon this ``approximation'' (actually assumption is more
    accurate) perform poorly for flows with strong curvature and flows in
    rotating fluids.  Consequently, results from Reynolds Stress Models (RSM)
    such as the Wilcox stress-$\omega$ model\cite{wilcox} and the
    SSG/LRR-$\omega$\cite{Cecora} have been somewhat encouraging, however they
    are still lacking when predicting separated flows.  This is likely due to
    poor progress in the development of a truly physics-based length scale
    equation.  Kantha\cite{Kantha} derives a generalized equation for this, and
    claims to show that current equations used to model the turbulent
    dissipation rate, $\varepsilon$, and turbulent frequency, $\omega$, are all
    subsets of a general form.  These equations are strongly based on empirical
    derivations, not physical ones, and are therefore a considerable hinderance
    to the success of a second-order closure model when applied to complex
    flows.  Thus, a suggestion to improve current RANS methods is to improve the
    equation used to determine the length scale and couple it with a RSM.
    
    The enstrophy equation in the k-$\zeta$ model developed by Robinson and
    Hassan\cite{Robinson} has potential, since it is has a stronger physical
    basis on vorticity fluctuations.  Unfortunately efforts to recast the
    k-$\zeta$ model as an RSM led to issues with convergence\cite{Xiao}, and
    consequently, poor results.  Nonetheless, the analysis by Tennekes and
    Lumley\cite{lumley} suggests that the energy transfer between large and
    small turbulent eddies is a vortex stretching process due to nonlinearity,
    suggesting that enstrophy is a critical component in turbulence.  Morinishi
    et. al. \cite{morinishi} gives a relationship between the turbulent
    dissipation, $\varepsilon$, and enstrophy, $\zeta$, thus it may be
    advantageous to replace the $\omega$ equation in the SSG/LRR-$\omega$ RSM or
    Wilcox stress-$\omega$ RSM, with the enstrophy equation defined in
    \cite{Xiao}.  At the very least, this is useful in determining if the
    convergence issues previously described can be overcome using a different
    reynolds stress tensor formulation, which does not necessarily contract to
    turbulent kinetic energy, k, equation presented in \cite{Robinson}.

  \item There are many ways to derive an expression for change in entropy.  From
    a classical thermodynamics perspective, this comes from the definition of
    Gibbs free energy, $G$
    \begin{equation}
      G = H - TS
      \label{gfe-def}
    \end{equation}
    Where $H$ is enthalpy, $T$ is temperature, and $S$ is entropy.  It is
    straightforward to see that the first and second laws of thermodynamics can
    be used to determine the change of entropy
    \begin{align}
      dQ &= dH - VdP \label{1st-law}\\
      dS &= \frac{dQ}{T} + dS_{i}= \frac{dH - Vdp}{T} + dS_{i}
      \label{2nd-law}
    \end{align}
    Where $dS_i$ is entropy generated from irreversible processes. Using the
    definition of the enthalpy \begin{equation}
      H = E + pV
      \label{enthalpy-def}
    \end{equation}
    \eref{2nd-law} can be expanded as
    \begin{equation}
      \begin{aligned}
	      dS &= \frac{dE + pdV + Vdp - Vdp}{T} + dS_i\\
	         &= \frac{dE + pdV}{T} + dS_i
      \end{aligned}
      \label{s-one-gas}
    \end{equation}
    While this is useful, this derivation is deficient as it does not directly
    account for multiple species in chemical non-equilibrium.  Revisiting Gibbs
    free energy for multiple species, if a system is considered for each
    isolated species then
    \begin{equation}
      G_s = G_s(p,T,N_s)
      \label{G-species-def}
    \end{equation}
    where $G_s$ is the Gibbs free energy of the isolated species $s$, and $N_s$
    is the number of particles of species $s$. Differentiating
    \eref{G-species-def} yields
    \begin{equation}
      dG_s = \left( \frac{\partial G_s}{\partial T} \right)_{(p,N_s)} dT
      + \left( \frac{\partial G_s}{\partial p} \right)_{(T,N_s)} dp
      + \left( \frac{\partial G_s}{\partial N_s} \right)_{(p,T)} dN_s
      \label{G-species-diff}
    \end{equation}
    Because $G = \sum\limits_{\substack{s=1}}^{N_{ns}}{G_s}$,
    \eref{G-species-diff} can be written as
    \begin{equation}
      dG = \left( \pd{G}{T} \right)_{(p,N_s)} dT
      + \left( \pd{G}{P} \right)_{(T,N_s)} dp
      + \sum\limits_{s=1}^{N_{ns}}{\left( \left( \pd{G}{N_s}
      \right)_{(p,T)} dN_s\right)}
      \label{G-diff}
    \end{equation}
    Introducing the concept of chemical potential
    \begin{equation}
      \mu_s = \left( \frac{\partial G}{\partial N_s} \right)_{p,T}
      \label{mu-def}
    \end{equation}
    and writing \eref{gfe-def} in differential form and substituting
    \eref{s-one-gas} yields
    \begin{equation}
      \begin{aligned}
        dG &= dH - SdT - TdS \\
	         &= dH - SdT - T\left( \frac{dH - Vdp}{T} + dS_i \right) \\
	         &= Vdp - SdT - TdS_i
      \end{aligned}
      \label{G-def-diff}
    \end{equation}
    It is clear that in comparing \eref{G-def-diff} and \eref{G-diff} that
    \begin{equation}
      \left( \pd{G}{T} \right)_{(p,N_s)} = -S, \quad
      \left( \pd{G}{P} \right)_{(T,N_s)} = V
      \label{G-derivs}
    \end{equation}
    and most importantly
    \begin{equation}
      dS_i = -\frac{1}{T}\sum\limits_{s=1}^{N_{ns}}{\mu_s dN_s}
      \label{si-def}
    \end{equation}
    Thus, it is now clear that the change of entropy is defined in two
    components: the entropy generated by the surroundings, $S_r$, and the
    entropy generated by irreversible chemical processes, $S_i$.  If
    \eref{G-def-diff} is expanded with $dS_i = 0$ and the definition of
    enthalpy in \eref{enthalpy-def} is used, the entropy from reversible
    processes is given
    by
    \begin{equation}
      dS_r = \frac{dE}{T} + \frac{pdV}{T}
      \label{sr-def}
    \end{equation}
    combining \erefs{si-def}{sr-def} the change of entropy expression is
    obtained
    \begin{equation}
      \boxed{dS = \frac{dE}{T} + \frac{pdV}{T} 
      -\frac{1}{T}\sum\limits_{s=1}^{N_{ns}}{\mu_s dN_s}}
      \label{ds-final}
    \end{equation}
    
    If two gases with the same pressure and temperature are mixed, then
    \eref{ds-final} shows that entropy must be generated, regardless of whether
    chemical reactions occur.  Since entropy of a system of gases is additive,
    \begin{equation}
      dS_{mix} = dS_1 + dS_2
      \label{s-additive}
    \end{equation}
    where $dS_{mix}$ is the change of entropy in the mixture, and $dS_{(1,2)}$
    are the changes in entropy of species being mixed.  If $E = E(T,p)$ and
    there is no change in temperature or pressure, $dE = 0$ for both species;
    however, the volume has changed.  For two intert gases mixing there will be
    production of entropy, since
    \begin{equation}
      dS_{mix} = \frac{p}{T}\left( dV_1 + dV_2 \right)
      \label{smix-change}
    \end{equation}
    Where $dV_1$ and $dV_2$ are the changes in volume for species 1 and 2,
    respectively.  Using the ideal gas law for the individual intert gases
    \begin{equation}
      p = \frac{nRT}{V}
      \label{ideal-gas}
    \end{equation}
    where $R$ is the universal gas constant and $n$ is the number of moles of
    the gas, \eref{smix-change} can be integrated to give the change in entropy
    when the gases are combined
    \begin{equation}
      \boxed{\begin{aligned}
        \Delta S_{mix} &= \int_{V_1}^{V_1+V_2}\frac{n_1 R}{V}dV
        + \int_{V_2}^{V_1+V_2}\frac{n_2 R}{V}dV \\
        \Delta S_{mix} &= R\left( n_1 \ln\left( \frac{V_1 + V_2}{V_1} \right) + n_2
        \ln\left( \frac{V_1 + V_2}{V_2} \right) \right)
      \end{aligned}}
      \label{delta-change-smix}
    \end{equation}
    The production of entropy is to be expected, intuitively,
    since the diffusion of one gas into the other happens spontaneously.  Each
    gas will expand into the new volume available after mixing, naturally, and
    it takes effort to separate them; hence $dS_{mix} > 0$.  Likewise, if
    chemical reactions occur, entropy will also be generated as dictated by
    \eref{si-def}.

    It should be noted that \eref{delta-change-smix} can also be derived from
    statistical mechanics.  Beginning with the Boltzmann relation
    \begin{equation}
      S = k\ln(\Omega)
      \label{boltzmann-relation}
    \end{equation}
    where $k$ is the boltzmann constant and $\Omega$ is the ``thermodynamic
    probability''.  In the boltzmann limit, which is the macro state containing
    the maximum number of microstates,
    \begin{equation}
      \ln(\Omega) = N\left( \ln \frac{\sum\limits_{j}{C_j e^{-\beta\varepsilon_j}}}{N}
      + 1\right) + \beta E
      \label{boltz-limit}
    \end{equation}
    where $C_j$ is a microstate container, $\varepsilon_j$ is a microstate
    energy, and $\beta$ is an undetermined multiplier.  Substituting
    \eref{boltz-limit} into \eref{boltzmann-relation}
    \begin{equation}
      S = Nk\left( \ln \frac{\sum\limits_{j}{C_j e^{-\beta\varepsilon_j}}}{N}
      + 1\right) + \beta E
      \label{s-stat-beta}
    \end{equation}
    Using \eref{s-one-gas} to solve for $\beta$, \eref{s-stat-beta} can be
    rewritten in terms of the the partition function, $Q$, as
    \begin{align}
      S &= Nk\left( \ln\frac{Q}{N} + 1 \right) + \frac{E}{T} \\
      Q &= \sum\limits_{j}{C_j e^{-\varepsilon_j/kT}}
      \label{S-Q}
    \end{align}
    Because the partition functions is the product of all the energy modes
    \begin{equation}
      Q = Q_{trans}Q_{rot}Q_{vib}Q_{elec}
      \label{Q-def}
    \end{equation}
    and the only volume dependence that $Q$ has is in the translational mode
    \begin{align}
      Q_{trans} &= VQ^{'}(T) \label{q-trans} \\
      Q &= VQ^{'}(T) \label{q-t}
    \end{align}
    where $Q^{'}(T)$ is used to denote a partition function's temperature
    dependence.  Substituting \eref{q-t} into \eref{S-Q} gives a relation for
    entropy with a clear dependence on $T$ and $V$
    \begin{equation}
      S = Nk\left( \ln\frac{V Q^{'}(T)}{N} + 1 \right) + \frac{E}{T}
      \label{S-Q'}
    \end{equation}
    Since the entropy change of the mixture is the sum of the individual gases'
    entropy changes, and there is no change in temperature or pressure when
    mixing occurs
    \begin{equation}
      \begin{aligned}
        \Delta S_{mix} &= 
        N_1 k\left( \ln \frac{(V_1+V_2)Q^{'}T}{N_1} - \ln
        \frac{V_1 Q^{'}T}{N_1} \right) \\
        &+ N_2 k\left( \ln \frac{(V_1+V_2)Q^{'}T}{N_2} - \ln
        \frac{V_2 Q^{'}T}{N_2} \right) \\
        \Delta S_{mix} &= 
        N_1 k\left( \ln \frac{(V_1+V_2)}{V_1} \right)
        + N_1 k\left( \ln \frac{(V_1+V_2)}{V_2} \right)
      \end{aligned}
      \label{S-stat-final}
    \end{equation}
    Since $R=N_A k$, where $N_A$ is avogadro's number and $n = N/N_A$,
    \eref{S-stat-final} is equivilent to \eref{delta-change-smix}.

  \item As stated previously, gibbs free energy is defined as
    \begin{equation}
      G = G(T,p,N_1,N_2,\dots,N_{ns}) = H - TS
      \label{gibbs-formal}
    \end{equation}
    Using the definition of $G$ in \eref{gibbs-formal}, it was previously shown that
    \begin{equation}
      \boxed{ S = -\left( \pd{G}{T} \right)_{(p,N_s)}}
      \label{S-def}
    \end{equation}
    Where $G$ and $S$ are defined as Gibbs free energy and entropy per mole of
    the mixture, respectively, and $N_s$ is the number of particles of species
    $i$ per mole of the mixture.  Likewise, enthalpy, $H$, and specific heat at
    constant pressure, $C_p$, per mole of mixture can be determined from
    \erefs{gibbs-formal}{S-def} as
    \begin{gather}
      \boxed{H = G - T\left( \pd{G}{T} \right)_{(p,N_s)}} \label{h-def} \\
      \boxed{C_p = \left(\pd{H}{T}\right)_{(p,N_s)} 
      = \left( \pd{G}{T} \right)_{(p,N_s)} - \left( \pd{G}{T} \right)_{(p,N_s)}
      - T\left( \pnd{G}{T}{2} \right)_{(p,N_s)}}
      \label{cp-def} \end{gather}
    These thermodynamic quantities can also be found for each species (per mole
    of mixture) by substituting $G_s$ in place of $G$, where $G_s$ is the Gibbs
    free energy of species $s$ per mole of the mixture, and
    \begin{equation}
      G = \sum\limits_{s=1}^{ns}{\chi_s G_s}
      \label{gs-sum}
    \end{equation}
    where $\chi_s$ is the mole fraction of species $s$.  Thus, if the variables
    $T$, $p$, $N_1$, \dots, $N_{ns}$ are known, Gibbs free energy can be used to
    derive all the thermodynamic quantities needed for a reacting gas solver.
    Likewise, if the mixture composition is not know, and needs to be determined
    at chemical equilibrium for a given $p$ and $T$, this can be done by free
    energy minimization.  Gordon and McBride\cite{mcbride} explain this in
    detail, and have developed a chemical equilibrium solver based on this
    principle.
    
    One other useful quantity that the Gibbs free energy provides is the
    reaction equilibrium constant in partial pressures, $K_p$ 
    \begin{gather}
      \boxed{K_p = 
        \prod_{i=1}^{ns}{p_i^{(\nu_i^{''}-\nu_i^{'})}} 
        = e^{-\Delta G /RT}} \label{kp-def} \\ 
      \boxed{\Delta G =
        \sum\limits_{i=1}^{ns}{\left(\nu_{i}^{''} - \nu_{i}^{'}\right) G_i}}
        \label{del-g-def} 
    \end{gather}
    where $\nu_i^{''}$ is the stoichiometric
    coefficient of the product species $i$, $\nu_i^{'}$ is the stoichiometric
    coefficient of the reactant species $i$, and $R$ is the universal gas
    constant.  Converting this to the more useful reaction equilibrium
    constant in concentrations, $K_c$, assuming the partial pressure is in
    units atm 
    \begin{equation} 
      \begin{gathered} 
        K_c = \left( \frac{101325}{RT}
        \right)^{(m^{''}-m^{'})} K_p \\
        m^{''} = 
        \sum\limits_{i=1}^{ns}{\nu_i^{''}}, \quad m^{'} =
        \sum\limits_{i=1}^{ns}{\nu_i^{'}} 
      \end{gathered} 
      \label{kc-def}
    \end{equation}
    This is a more useful quantity, as it can be used to obtain
    the backward reaction rate constant, $k_b$, based on the forward reaction
    rate coefficient,$k_f$, where 
    \begin{equation} 
      k_b = \frac{k_f}{K_c}
      \label{kb-def} 
    \end{equation}

\bibliography{hassan-prelim}
\bibliographystyle{plain}

\end{enumerate}

\end{document}


