%        File: report.tex
%     Created: Fri Mar 04 10:00 AM 2016 E
% Last Change: Fri Mar 04 10:00 AM 2016 E
%
\documentclass[a4paper]{report}

%\usepackage[a4paper, margin=1in]{geometry}

\usepackage{listings}
\usepackage{mathtools}
\usepackage{amsmath}
\usepackage{amssymb}
\usepackage{newlfont}
\usepackage{caption}
\usepackage{subcaption}
\usepackage{titlesec}
\usepackage{empheq}
\usepackage{pdfpages}
\usepackage{enumitem}
\usepackage{bm}

\newcommand{\eref}[1]{Eq.~(\ref{#1})}
\newcommand{\erefs}[2]{Eq.s~(\ref{#1}-\ref{#2})}
\newcommand{\dint}[1]{\int_{\Omega_i}{#1 d\Omega}}
\newcommand{\sint}[1]{\int_{\Gamma_{ij}}{#1 d\Gamma}}
\newcommand{\average}[1]{\ensuremath{\{#1\}} }
\newcommand{\jump}[1]{\ensuremath{[\![#1]\!]} }
\newcommand{\vbasis}{\boldsymbol{\tau}}
\newcommand{\glift}{\boldsymbol{\delta}}
\newcommand{\llift}{\boldsymbol{\delta_l}}
\newcommand{\unitn}{\mathbf{n}}
\newcommand{\mm}{\mathbf{M}}
\newcommand{\vr}{\mathbf{r}}

%\newcommand{\d}{\partial}

\title{MAE 766 Written Preliminary Exam \#1}

\author{ Kyle B. Thompson }

\begin{document}
\maketitle

\section{Introduction}
Discontinous Galerkin (DG) methods have seen significant development in the last
three decades.  Incorporating aspects of finite-volume and finite-element
methods, DG schemes offer a compact way of achieving higher order accuracy by
solving for polynomial basis coefficients that are defined uniquely on each
element.  Discontinuities at the element boundaries are reconciled via ``flux
functions'', which provide the means that information is propogated between
elements.  This is a significant advantage when solving hyperbolic partial
differential equations (PDEs) with discontinuities in the solution, since these
discontinuities can be handled properly by flux functions using an upwind
mechanism.  Unfortunately, DG method becomes problematic when dealing with
elliptic PDEs.  Since there is not an upwinding mechanism for the diffusion
operator, a central difference seems to be an intuitive choice; however, this
has been shown to result in an inconsistent scheme. The central difference flux
does not capture jumps in the solution, only in the solution gradients.  To
overcome this issue, Bassi and Rebay developed a flux scheme to account for
jumps in the solution via ``local lifting operators'', which ``lift'' the
solution to account for $C_0$ continuity.

\section{Problem Definition and Method of Solution}

Potential theory states that for incompressible flows that are irrotational, the
velocity potential function of a flow is governed by the Laplace Equation
\begin{equation}
  \Delta \phi = 0
  \label{laplace-equation}
\end{equation}
In two space dimensions this can be written out explicitly as
\begin{equation}
  \frac{\partial^2 \phi}{\partial x^2} 
  + \frac{\partial^2 \phi}{\partial y^2} = 0
  \label{2d-laplace}
\end{equation}
and the velocity is equivalent to the gradient of the potential function
\begin{equation}
  u = \frac{\partial \phi}{\partial x}, \qquad v = \frac{\partial \phi}{\partial y}
  \label{uv-def}
\end{equation}
The second scheme derived by Bassi and Rebay (BR2) solves this problem by
rewriting \eref{laplace-equation} as a first order system to be solved 
\begin{align}
  \nabla \cdot q &= 0 \\
  q - \nabla \phi &= 0
  \label{first-order-sys}
\end{align}
with $q$ being a vector auxilliary variable with dimensionality of the problem.
We proceed by defining the solution $\phi$ as a polynomial \begin{equation}
  \phi = \sum{\phi_i B_i}
  \label{fem-soln}
\end{equation}
where $\phi_i$ are the basis weights for the basis set $B_i$.  The solution can
now be iterpolated anywhere in the computational domain, and the basis set used
here is a linear finite-element basis based on barycentric coordinates
\begin{equation}
  \begin{gathered}
    B_i = \frac{a_i x + b_i y + c_i}{D} \\
    a_i = y_j-y_k, \quad 
    b_i = -(x_j - x_k), \quad
    c_i = x_j y_k - x_k y_j  \\
    D = c_1 + c_2 + c_2
  \end{gathered}
  \label{basis}
\end{equation}
where the subscripts $_i$, $_j$, and $_k$ denote different nodes around the
triangle. This is a nodal basis, and is useful in that the exact integrals of
the basis functions are given by
\begin{align}
  \dint{B_1^{m}B_2^{n}B_3^{l}} &= D \frac{m!\ n!\ l!}{(m+n+l+2)!}
  \label{dom-integral-def} \\
  \sint{B_1^{m}B_2^{n}} &= L \frac{m!\ n!}{(m+n+1)!}
  \label{edge-integral-def}
\end{align}
Where $\Omega_i$ denotes the domain integral over the element $i$, and
$\Gamma_{ij}$ denotes the contour integral over the interface or boundary of an
element.
By introducing the ``lift operator'' concept and recasting
\eref{first-order-sys} in integral form, it is possible to arrive at the
``primal formulation'' that solves the system in a single equation as a function
of the solution basis weights and the lifting operators.  The first Bassi and
Rebay scheme (BR1) defined a ``global lifting operator'', $\glift$, as
\begin{equation}
  \dint{\glift \cdot \vbasis} + \sum_{j=1}^3\sint{\frac{1}{2} \jump{\phi}
  \vbasis \cdot \unitn} = 0
  \label{global-lift-def}
\end{equation}
where $\vbasis$ is an arbitrary basis function with the dimensionality of the
problem. However, using this global lifting operator was found to be
non-compact, and unstable for purely elliptic problems, such as the Laplace
equation. The BR2 scheme overcomes both of these issues by defining a ``local
lifting operator'', $\llift$, defined at each element interface by
\begin{equation}
  \begin{gathered}
    \dint{\llift \cdot \vbasis} + \eta\sint{\frac{1}{2} \jump{\phi} \vbasis
    \cdot \unitn} = 0 \\
    \jump{\phi} = \phi_i - \phi_j
  \end{gathered}
  \label{local-lift-def}
\end{equation}
Where $\eta$ is a stability parameter and $\jump{\cdot}$ is a jump operator.  It
has been proven that $\eta \geq N_{edges}$ results in a stable scheme.
Additionally, since $\vbasis$ is an arbitrary basis, the global lifting operator
can be related to the local lifting operators by summing the local lift
contributions
\begin{equation}
  \glift = \sum_{e=1}^3{\llift^{(e)}}
  \label{local-global-relation}
\end{equation}
With this, the primal form is given by
\begin{equation}
  \dint{(\nabla \phi - \glift)\cdot \nabla B} -
  \sum_{j=1}^3\sint{\average{(\nabla \phi - \llift)\cdot\unitn}B} = 0
  \label{primal-form}
\end{equation}
where $\average{\cdot}$ is an average operator.

\section{Implementation}
Since the Laplace equation is a linear problem, it is possible to construct a
global system an solve for all unknowns simultaneously.  This is reminscient of
the continuous Galerkin (CG) method, where a stiffness matrix was formed for the left
hand side and a load vector for the right hand side.  In the Discontinuous Galerkin
formulation, however, continuity is not enforced across elements; therefore, the
number of unknowns going from CG(P1) to DG(P1) increases from the number of
points in the mesh to the number of elements times each elements' verticies.
Additionally, if the system is to be solved implicitly, each local lifting
operator $\llift$ must be solved for as well, increasing the total number of
equations to $3(N_{elem})+N_{faces}$ at a minimum.  This drastically increases
the problem size and complexity for an implicit solver, since a sparsity pattern
must be exploited to formulate an efficient linear solver.

An alternative to solving the system implicitly is to explicitly evolve the
solution in pseudo-time to steady state.  Adding a time derivative to
\eref{primal-form} gives
\begin{equation}
  \dint{\frac{\partial \phi}{\partial t}B} +
  \dint{(\nabla \phi - \glift)\cdot \nabla B} -
  \sum_{j=1}^3\sint{\average{(\nabla \phi - \llift)\cdot\unitn}B} = 0
  \label{primal-w-time}
\end{equation}
This results in a much simpler problem to solve, and the notation can be
contracted to
\begin{equation}
  \frac{\partial \phi}{\partial t} = -\mm^{-1} \vr
  \label{simple-explicit}
\end{equation}
where $\vr$ is the residual vector and $\mm$ is the block diagonal mass matrix.
If the boundaries are not time-dependent, then we recover the solution to the
Laplace equation upon convergence, since $\vr= 0$.  The mass matrix components can be
easily computed using \erefs{dom-integral-def}{edge-integral-def}, and the
system can be inverted by hand to signifcantly save computational time. The time
integration is done via Runge-Kutta and the time step can be computed locally or
globally based on a CFL prescribed.  Thus, this explicit DG scheme can be
evolved from an initial state to the solution of the laplace equation by
iteratively updating each degree of freedom until the residual is less than a
prescribed tolerance, with the local lift operators $\llift$ simply computed as
an auxilliary variable.

\section{Numerical Results}
The case being solved is a channel with a circular bump.  Figure ??? shows 
the Neumann boundary conditions prescribed, where flow in and out of the
channel is the freestream velocity (1,0), and no flux is permitted through the
channel walls.

  %\frac{\phi_{i}^{(n+1)}}{\Delta t}

\end{document}
