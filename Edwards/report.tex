%        File: report.tex
%     Created: Thu Feb 18 01:00 PM 2016 E
% Last Change: Thu Feb 18 01:00 PM 2016 E
%
\documentclass[a4paper]{report}

\usepackage{mathtools}
\usepackage{amsmath}
\usepackage{amssymb}
\usepackage{newlfont}
\usepackage{caption}
\usepackage{subcaption}
\usepackage{titlesec}
\usepackage{empheq}

\newcommand*\widefbox[1]{\fbox{\hspace{2em}#1\hspace{2em}}}

\newcommand{\eref}[1]{Eq.~(\ref{#1})}
\newcommand{\erefs}[2]{Eq.s~(\ref{#1}-\ref{#2})}
\newcommand{\sk}{\dot{s}_k}
\newcommand{\skg}[1]{{\dot{s}_{#1}}^{(g)}}
\newcommand{\sks}[1]{{\dot{s}_{#1}}^{(s)}}
\newcommand{\kf}[1]{k_{f,#1}}
\newcommand{\kb}[1]{k_{b,#1}}
\newcommand{\kcf}[1]{\frac{k_{f,#1}}{k_{c,#1}}}
\newcommand{\cg}[1]{{C_{#1}^{(g)}}}
\newcommand{\cs}[1]{{C_{#1}^{(s)}}}


\title{MAE 770 Written Preliminary Exam \#1}

\author{ Kyle B. Thompson }

\begin{document}
\maketitle
\begin{enumerate}
  \item The units can be determined from the left hand side of the equation
    \begin{equation}
      \rho D_k \frac{\partial Y_k^{(g)}}{\partial x_j} \cdot \hat{n}_j = M_k \sk
      \label{bc posed}
    \end{equation}
  Using the following nomenclature:
  \begin{tabbing}
    XXXXXXXXX \= \kill% this line sets tab stop
    $m$ \> mass \\
    $N$ \> number of molecules \\
    $l$ \> length \\
    $t$ \> time
  \end{tabbing}
  The units of the variables on the left hand side of \ref{bc posed} are
  \begin{align*}
    \rho &= \frac{m}{l^3} \\
    D_k &= \frac{l^2}{t} \\
    x_j &= l
  \end{align*}
  The mass fraction, $Y_k^{(g)}$, is unitless, as well as the unit normal vector,
  $\hat{n}_j$. Since the unit of molecular weight are, $M_k = m/N$, we can
  rearrange \eref{bc posed} and solve for the units of $\sk$
  \begin{equation}
    \label{units sdot}
    \begin{aligned}
      \sk &= \frac{\rho D_k}{M_k}\frac{\partial Y_k^{(g)}}{\partial x_j}\cdot \hat{n}_j \\
      &= \frac{m}{l^3} \frac{N}{m} \frac{l^2}{t}\frac{1}{l} \\
      &= \frac{N}{l^2 t}
    \end{aligned}
  \end{equation}
  Expressing surface coverage, $\Theta_i$, as the fraction of sites occupied by
  species $i$ implies that it is unitless.  The concentration of
  surface-absorbed species, $C_k$, has units $N/l^2$ and must be equivalent to
  the product of surface coverage and surface site density, $\Gamma$; therefore,
  the units of surface site density must be $N/l^2$.  Thus, the units for $\sk$
  and $\Gamma$, expressed in the same basis as the problem was posed, are
  \[
    \boxed{\sk = \frac{moles}{area^2-sec}}
  \]
  \[
    \boxed{\Gamma = \frac{moles}{area^2}}
  \]

\item
  The formulation of the source terms $\sk$ for $k=$ gas species and $k=$
  surface species can be expressed as
  \begin{equation}
    \label{sdot form}
    \begin{gathered}
      \sk = \sum_{r=1}^{N_r}\left[(\nu_{k,r}^{''} - \nu_{k,r}^{'})
      (R_{f,r} - R_{b,r})\right] \\
      R_{f,r} = k_{f,r}\prod_{i=1}^{N_k}(C_i^{\nu_{i,r}^{'}}), \quad
      R_{b,r} = k_{b,r}\prod_{i=1}^{N_k}(C_i^{\nu_{i,r}^{''}})
    \end{gathered}
  \end{equation}
  where $\nu_{k,r}^{'}$ is the stoichometric coefficient for species $k$ in
  reaction $r$ as a reactant in the forward reaction, and $\nu_{k,r}^{''}$ is
  for that species as a product in the forward reaction.  $N_k$ denotes the
  number of species and $N_r$ denotes the number of reactions.  Additionally,
  the backward rate coefficient $k_{b,r}$ can be determined from the forward
  rate coefficient $k_{f,r}$ and the equilibrium constant $k_{c,r}$.  This is much
  better, since the law of mass action enables the determination of $k_{c,r}$ as
  a function of species concentrations, $C_i$, resulting in $\sk =
  \sk(k_{f,r},C_1,\dots,C_{N_k})$. Thus, using \eref{sdot form} the $\sk$ source
  terms can be expressed as
  \begin{empheq}[box=\boxed]{align}
    \skg{H_2}  &= -\left( \kf{1} \cg{H_2} - \kcf{1} {\cs{H}}^2 \right) \\
    \skg{O_2}  &= -\left( \kf{2} \cg{O_2} - \kcf{2} {\cs{O}}^2 \right) \\
    \skg{H_2O} &= \kf{6} \cs{H_2O} - \kcf{6} \cg{H_2O} \\
    \begin{split}
      \sks{H}    &= 2 \left( \kf{1} \cg{H_2} - \kcf{1} {\cs{H}}^2 \right) \\
      &-\left( \kf{3} \cs{H} \cs{O} - \kcf{3} \cs{OH} \right) \\
      &-\left( \kf{4} \cs{H} \cs{OH} - \kcf{4} \cs{H_2O} \right)
    \end{split} \\
    \begin{split}
      \sks{O} &= 2 \left( \kf{2} \cg{O_2} - \kcf{2} {\cs{O}}^2 \right) \\
      &-\left( \kf{3} \cs{H} \cs{O} - \kcf{3} \cs{OH} \right) \\
      &+\left( \kf{5} {\cs{OH}}^2 - \kcf{5} \cs{H_2O} \cs{O} \right)
    \end{split} \\
    \begin{split}
      \sks{OH} &= 2 \left( \kf{3} \cs{H} \cs{O} - \kcf{3} \cs{HO} \right) \\
      &-\left( \kf{4} \cs{H} \cs{OH} - \kcf{4} \cs{H_2O} \right) \\
      &-2\left( \kf{5} {\cs{OH}}^2 - \kcf{5} \cs{H_2O} \cs{O} \right)
    \end{split} \\
    \begin{split}
      \sks{H_2O} &= \left( \kf{4} \cs{H} \cs{OH} - \kcf{4} \cs{H_2O} \right) \\
      &+\left( \kf{5} {\cs{OH}}^2 - \kcf{5} \cs{H_2O} \cs{O} \right) \\
      &-\left( \kf{6} \cs{H_2O} - \kcf{6} \cg{H_2O} \right)
    \end{split}
    \label{source terms}
  \end{empheq}

\item Determining the units for the forward and backward rate coefficients,
  $k_f$ and $k_b$, begins with looking at the units of $\sk$.  Since the source
  term was found to have units of $moles/(area^2$-$sec)$, in the generic sense, it
  would have proper units of $mol/(cm^2$-$s)$ using a $cm$, $mol$, $s$ basis.
  The units of $k_f$ and $k_b$ will vary for each reaction, but must ultimately
  conform to the fact that $R_{f,r}$ and $R_{b,r}$ in \eref{sdot form} must have
  units of $mol/(cm^2$-$s)$.  For the first reaction
  \begin{equation}
    \kf{1} = 1.9e19 \quad \frac{cm}{s}
    \label{kf_units}
  \end{equation}
  which satisfies
  \begin{equation}
    R_{f,1} = k_{f,1}\cg{H_2} = \frac{cm}{s}\frac{mol}{cm^3} =
    \frac{mol}{cm^2\text{-}s}
    \label{R_units}
  \end{equation}
 Much more generally, the units for $k_f$ can be expressed as
  \begin{equation}
    \kf{r} = \frac{mol}{(cm^2)(s)\prod_{i=1}^{N_k}\left(
    \left( \frac{mol}{cm^3} \right)^{\nu_{i,r}^{'(g)}} \left( \frac{mol}{cm^2}
    \right)^{\nu_{i,r}^{'(s)}} \right)}
    \label{kf-unit-formula}
  \end{equation}
  where $\nu_{i,r}^{'(g)}$ is the stoichiometric coefficient of gas-phase
  reactant species $i$ in reaction $r$, and $\nu_{i,r}^{'(s)}$ is the
  stoichiometric coefficient of surface reactant species $i$ in reaction $r$.
  Since $R_{b,r}$ must also have units of $mol/(cm$-$s)$, the same reasoning can
  be used to derive
  \begin{equation}
    \kb{r} = \frac{mol}{(cm^2)(s)\prod_{i=1}^{N_k}\left(
    \left( \frac{mol}{cm^3} \right)^{\nu_{i,r}^{''(g)}} \left( \frac{mol}{cm^2}
    \right)^{\nu_{i,r}^{''(s)}} \right)}
    \label{kb-unit-formula}
  \end{equation}
  where $\nu_{i,r}^{''(g)}$ and $\nu_{i,r}^{''(s)}$ are the are the
  stoichiometric coefficients of gas-phase and surface product species $i$ in
  reaction $r$, respectively.  Using \erefs{kf-unit-formula}{kb-unit-formula},
  the units of $\kf{r}$ and $\kb{r}$ (using a $cm$, $mol$, $s$ basis) are
  \begin{align}
    \begin{alignedat}{2}
      \kf{1} &= 1.9e19 && \quad \frac{cm}{s} \\
      \kf{2} &= 3e21   && \quad \frac{cm}{s} \\
      \kf{3} &= 6e18   && \quad \frac{cm^2}{mol\text{-}s} \\
      \kf{4} &= 1e7    && \quad \frac{cm^2}{mol\text{-}s} \\
      \kf{5} &= 2.4e24 && \quad \frac{cm^2}{mol\text{-}s} \\
      \kf{6} &= 1e5    && \quad \frac{1}{s}
    \end{alignedat} \qquad\qquad
    \begin{alignedat}{2}
      \kb{1} &= \kcf{1} \, && \frac{cm^2}{mol\text{-}s} \\
      \kb{2} &= \kcf{2} \, && \frac{cm^2}{mol\text{-}s} \\
      \kb{3} &= \kcf{3} \, && \frac{1}{s} \\
      \kb{4} &= \kcf{4} \, && \frac{1}{s} \\
      \kb{5} &= \kcf{5} \, && \frac{cm^2}{mol\text{-}s} \\
      \kb{6} &= \kcf{6} \, && \frac{cm}{s}
    \end{alignedat}
    \label{kf-kb-cgs}
  \end{align}
  by the law of mass action, the equilibrium constant can be defined as
  \begin{equation}
    k_{c,r} = \frac{\prod_{i=1}^{N_k}(C_i^{\nu_{i,r}^{''}})}
    {\prod_{i=1}^{N_k}(C_i^{\nu_{i,r}^{'}})}
    \label{kc-def}
  \end{equation}
  Converting $\kf{r}$ in \eref{kf-kb-cgs} to an SI basis ($m$, $kmol$, $s$),
  using $m = 100\, cm$ and $kmol = 1000\, mol$, yields
  \begin{equation}
    \boxed{\begin{alignedat}{2}
      \kf{1} &= 1.9e17 && \quad \frac{m}{s} \\
      \kf{2} &= 3e19   && \quad \frac{m}{s} \\
      \kf{3} &= 6e17   && \quad \frac{m^2}{kmol\text{-}s} \\
      \kf{4} &= 1e6    && \quad \frac{m^2}{kmol\text{-}s} \\
      \kf{5} &= 2.4e23 && \quad \frac{m^2}{kmol\text{-}s} \\
      \kf{6} &= 1e5    && \quad \frac{1}{s}
    \end{alignedat}}
  \end{equation}
  Using this and \eref{kc-def}, $\kb{r}$ can also be defined in an SI basis
  \begin{equation}
    \boxed{\begin{alignedat}{2}
      \kb{1} &= 1.9e17 \left(\frac{\cg{H_2}}{\cs{H}^2}\right) 
                \quad && \frac{m^2}{kmol\text{-}s} \\
      \kb{2} &= 3e19   \left(\frac{\cg{O_2}}{\cs{O}^2}\right)
                \quad && \frac{m^2}{kmol\text{-}s} \\
      \kb{3} &= 6e17   \left(\frac{\cs{H}\cs{O}}{\cs{OH}}\right)
                \quad && \frac{1}{s} \\
      \kb{4} &= 1e6    \left(\frac{\cs{H}\cs{OH}}{\cs{H_2O}}\right)
                \quad && \frac{1}{s} \\
      \kb{5} &= 2.4e23 \left(\frac{\cs{OH}^2}{\cs{H_2O}\cs{O}}\right)
                \quad && \frac{m^2}{kmol\text{-}s} \\
      \kb{6} &= 1e5    \left(\frac{\cs{H_2O}}{\cg{H_2O}}\right)
                \quad && \frac{m}{s}
    \end{alignedat}}
    \label{kb-final}
  \end{equation}
  It should be noted that the species concentrations $C_{i}^{(g,s)}$ must be in
  an SI basis for the units in \eref{kb-final} to be the correct units of
  $\kb{r}$.  If the concentrations are in a another basis, then another
  conversion will be necessary.

\end{enumerate}
\end{document}
